\section*{Introduction:}\label{sec:introduction}
Singapore, a frontrunner in sustainable urban development, grapples with a crucial data gap: the lack of a dedicated Wireless Sensor Network (WSN) to monitor harmful gas (CO\textsubscript{2})z emissions and their intricate link to temperature fluctuations across diverse urban environments. This absence of comprehensive data impedes our ability to accurately track progress towards ambitious environmental targets and formulate informed policies for critical issues like CO\textsubscript{2} reduction and urban heat island mitigation.

\section*{Problem Statement}\label{sec:problem_statement}
The lack of comprehensive environmental monitoring poses a significant challenge within Singapore. Currently, there is a dearth of real-time data on CO\textsubscript{2} levels and temperature, hindering informed decision-making and sustainable practices in the country. This data gap inhibits the establishment of correlations between environmental factors and their impact on campus well-being and sustainability. Consequently, there is a critical need to develop and deploy a scalable and energy-efficient mesh network utilizing efficient communication protocols to address this challenge. 

\subsection{Objectives}\label{sec:objectives}

By providing real-time environmental data, this project aims to enable:

\begin{itemize}
    \item Identification of correlations between CO\textsubscript{2} levels, temperature, and campus environmental conditions.
    \item Evaluation of the performance and suitability of LoRa and ESP-Now protocols in comparison to traditional mesh algorithms, addressing concerns regarding data latency, reliability, network scalability, and reach.
    \item Provision of actionable insights to policymakers and stakeholders, facilitating evidence-based strategies for enhancing environmental sustainability within NYP Campus and contributing to broader sustainability initiatives in Singapore.
\end{itemize}

Through the resolution of this pressing problem, the project endeavors to pave the way for data-driven decision-making and sustainable management practices within NYP Campus, aligning with Singapore's commitment to environmental stewardship.

% \section*{Problem Statement}\label{sec:problem_statement}
% This project tackles this pressing challenge by proposing the development and deployment of a scalable and energy-efficient mesh network utilizing LoRa and ESP-Now protocols in Nanyang Polytechnic (NYP) Campus. Through this network, we aim to collect and analyze real-time CO\textsubscript{2} and temperature data, enabling us to achieve three key objectives:
% 
% \begin{itemize}
%     \item Establish robust correlations between these environmental factors,
%     \item Evaluate the performance and suitability of LoRa and ESP-Now protocols compared to established mesh algorithms, and
%     \item Deliver valuable insights to policymakers and stakeholders, empowering them to develop data-driven strategies for a more sustainable urban future.
% \end{itemize}
% 
% By addressing this data gap and providing actionable insights, this project aspires to contribute significantly to Singapore's journey towards environmental sustainability.
% 
% \subsection*{Focus Areas}\label{sec:focus_areas}
% We will focus on key factors like:
% \begin{itemize}
%     \item \textbf{Data latency and reliability:} Ensure timely and accurate transmission of environmental data.
%     \item \textbf{Network scalability and reach:} Ability to handle a large number of nodes and cover the desired area effectively.
%     \item \textbf{Energy efficiency:} Minimize power consumption of sensor nodes for extended lifespan and network sustainability.
%     \item \textbf{Cost-effectiveness:} Consider hardware, deployment, and operational costs for a sustainable solution.
% \end{itemize}
