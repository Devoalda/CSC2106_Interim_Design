
\section*{Literature Review}

\subsection*{Routing Protocol for Low Power Lossy Networks (RPL)}

Designing a robust and efficient mesh network for real-time environmental monitoring poses unique challenges due to the resource-constrained nature of wireless sensor networks (WSNs). The Routing Protocol for Low-Power and Lossy Networks (RPL), a proactive routing protocol \cite{rechache_study_2021}, emerges as a promising candidate, addressing these challenges head-on with its standardized approach, dynamic routing capabilities, and energy-efficient design \cite{alexander_rpl_2012, Kharrufa2019RPL-Based}. Notably, RPL aligns well with our project's key objectives of scalability, reliability, and sustainability.

One of RPL's critical strengths lies in its ability to dynamically adapt to network changes. An adaptation mentioned in \cite{kharrufa_dynamic_2017}, D-RPL, optimises multihop routing, guaranteeing reliable data collection even when nodes fail or move, or new ones are deployed, a crucial feature for real-time environmental monitoring where network conditions are constantly shifting.

Furthermore, RPL prioritizes energy efficiency \cite{info8040124}. By employing energy-aware path selection and minimizing control overhead, RPL extends the lifespan of individual nodes and the overall network, a significant advantage in resource-constrained environments.

Finally, RPL exhibits excellent scalability, making it suitable for large-scale WSNs like the one envisioned for the application. Its hierarchical design effectively handles complex network structures, ensuring efficient data routing even as the network grows in size and complexity.

While RPL boasts significant advantages, careful consideration must be given to its limitations before implementation. Security measures require meticulous implementation to address potential vulnerabilities in resource-constrained environments \cite{arena_evaluating_2020, mayzaud_distributed_2017, 8998289}. Additionally, RPL's dynamic nature can introduce complexity and overhead compared to simpler protocols, demanding thoughtful analysis and optimization.

Balancing RPL's strengths with its limitations is crucial. Its dynamic routing, energy efficiency, and scalability hold immense potential for achieving our project's goals. Leveraging RPL's features, particularly its dynamic routing and multipathing capabilities, promises timely data collection even in dynamic network environments, contributing to a successful monitoring application. Further investigation and experimentation are necessary to determine the optimal configuration and potential adaptations required for RPL to seamlessly integrate into our specific project and effectively meet the desired outcomes.
