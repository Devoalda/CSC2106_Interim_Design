% XY LR %
\subsection*{Optimised Link State Routing (OLSR)}

In networks using the Optimised Link State Routing (OLSR) routing protocol, each node selects a neighbor that is directly reachable (one hop away) when transmitting data packets. These selected neighbors form a set known as Multi-Point Relay (MPR). Only nodes in the MPR set are responsible for forwarding messages, ensuring efficient packet transmission\cite{Kakade2017Performance} \cite{OLSR_IETF}. MPR nodes are chosen to cover as many other nodes as possible within one hop, optimizing network coverage\cite{Ahn2014-ii}. 
In an island wide C02  Monitoring deployment using OLSR, the scalability of OLSR would highly benefit the how well it is suited for large scale networks with its efficient routing table management and can reliably forward packets . 
With the nature of OLSR proactive routing can also help minimise network throughput by reducing routing overhead when sending packets. And due to this optimised routing energy consumption is reduced for nodes not involved in message forwarding and processing\cite{Guo_2011}.
OLSR is also able to adapt to node failure and recalculate routes between nodes to ensure message forwarding reliability which would highly benefit large scale deployments in an urban area like Singapore for effective and reliable monitoring.
Power consumption is one critical factor that OLSR adapts due to its routing mechanism can adapt to prolong battery life and illeviate the battery consumption\cite{Jubair_2019}. This is done by maintaing a network topology of nodes to reduce power needed for route discovery and this is also maintained in a routing table in each node.
While security is not inherent in OLSR, encryption and firewall and access control can be used to reduced risk of unauthorised access and network abuse.
Overall, Optimized Link State Routing protocol offers several advantages such as scalability, reliability, and low latency, its suitability for large-scale deployment in IoT CO2 monitoring applications.

% Pxy LR References %

@article{Kakade2017Performance,title={Performance Analysis of OLSR to Consider Link Quality of OLSR-ETX/MD/ML in Wireless Mesh Networks},author={S. Kakade and P. Khanagoudar},journal={2017 2nd International Conference on Computational Systems and Information Technology for Sustainable Solution (CSITSS)},year={2017},pages={1-10},doi={10.1109/CSITSS.2017.8447791}}

@ARTICLE{OLSR_IETF,
  author={T. Clausen Ed., and P. Jacquet Ed. },
  journal={IEEE Sensors Journal}, 
  title={Optimized Link State Routing Protocol (OLSR). IETF}, 
  year={2003},
  pages={4},
  abstract={This document describes the Optimized Link State Routing (OLSR)
   protocol for mobile ad hoc networks.  The protocol is an optimization
   of the classical link state algorithm tailored to the requirements of
   a mobile wireless LAN.  The key concept used in the protocol is that
   of multipoint relays (MPRs).  MPRs are selected nodes which forward
   broadcast messages during the flooding process.  This technique
   substantially reduces the message overhead as compared to a classical
   flooding mechanism, where every node retransmits each message when it
   receives the first copy of the message.  In OLSR, link state
   information is generated only by nodes elected as MPRs.  Thus, a
   second optimization is achieved by minimizing the number of control
   messages flooded in the network.  As a third optimization, an MPR
   node may chose to report only links between itself and its MPR
   selectors.  Hence, as contrary to the classic link state algorithm,
   partial link state information is distributed in the network.  This
   information is then used for route calculation.  OLSR provides
   optimal routes (in terms of number of hops).  The protocol is
   particularly suitable for large and dense networks as the technique
   of MPRs works well in this context.}
}

@article{Ahn2014,
  title = {Multipoint relay selection for robust broadcast in ad hoc networks},
  volume = {17},
  ISSN = {1570-8705},
  url = {http://dx.doi.org/10.1016/j.adhoc.2014.01.007},
  DOI = {10.1016/j.adhoc.2014.01.007},
  journal = {Ad Hoc Networks},
  publisher = {Elsevier BV},
  author = {Ahn,  Ji Hyoung and Lee,  Tae-Jin},
  year = {2014},
  month = jun,
  pages = {82–97}
}

@article{Guo_2011, title={Energy aware proactive optimized link state routing in mobile ad-hoc networks}, volume={35}, ISSN={0307-904X}, url={http://dx.doi.org/10.1016/J.APM.2011.03.056}, DOI={10.1016/j.apm.2011.03.056}, number={10}, journal={Applied Mathematical Modelling}, publisher={Elsevier BV}, author={Guo, Zhihao and Malakooti, Shahdi and Sheikh, Shaya and Al-Najjar, Camelia and Lehman, Matthew and Malakooti, Behnam}, year={2011}, month=oct, pages={4715–4729} }

@article{Jubair_2019, title={Bat Optimized Link State Routing Protocol for Energy-Aware Mobile Ad-Hoc Networks}, volume={11}, ISSN={2073-8994}, url={http://dx.doi.org/10.3390/sym11111409}, DOI={10.3390/sym11111409}, number={11}, journal={Symmetry}, publisher={MDPI AG}, author={Jubair, Mohammed Ahmed and Mostafa, Salama A. and Muniyandi, Ravie Chandren and Mahdin, Hairulnizam and Mustapha, Aida and Hassan, Mustafa Hamid and Mahmoud, Moamin A. and Al-Jawhar, Yasir Amer and Al-Khaleefa, Ahmed Salih and Mahmood, Ahmed Jubair}, year={2019}, month=nov, pages={1409} }