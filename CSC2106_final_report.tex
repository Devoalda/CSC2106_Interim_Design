\documentclass[
  conference, 
%   draft % remove this for final version
]{IEEEtran}
\IEEEoverridecommandlockouts
% The preceding line is only needed to identify funding in the first footnote. If that is unneeded, please comment it out.
%\usepackage{cite}
\usepackage{amsmath,amssymb,amsfonts}
\usepackage{algorithmic}

\usepackage{graphicx}
\graphicspath{{Figures/}{./}}

\usepackage{tabularx}

\usepackage{import}

\usepackage{textcomp}
\usepackage{xcolor}
\usepackage[section]{placeins}
\usepackage{float}

\usepackage[
backend=biber,
style=ieee,
]{biblatex}
\addbibresource{references.bib}
\def\BibTeX{{\rm B\kern-.05em{\sc i\kern-.025em b}\kern-.08em
    T\kern-.1667em\lower.7ex\hbox{E}\kern-.125emX}}
\begin{document}

\title{T24: Dynamic Mesh Wireless Sensor Collection Network}

\author{

\IEEEauthorblockN{1\textsuperscript{st} Woon Jun Wei}
\IEEEauthorblockA{\textit{Computing Science Joint Degree} \\
Programme, Singapore Institute\\
of Technology - University of\\
Glasgow,\\
2837269W@student.gla.ac.uk}
\and

\IEEEauthorblockN{2\textsuperscript{nd} Low Hong Sheng Jovian }
\IEEEauthorblockA{\textit{Computing Science Joint Degree} \\
Programme, Singapore Institute\\
of Technology - University of\\
Glasgow,\\
2837402L@student.gla.ac.uk}
\and

\IEEEauthorblockN{3\textsuperscript{rd} Benjamin Loh Choon How }
\IEEEauthorblockA{\textit{Computing Science Joint Degree} \\
Programme, Singapore Institute\\
of Technology - University of\\
Glasgow,\\
2837341L@student.gla.ac.uk}
\and

\IEEEauthorblockN{4\textsuperscript{th} Wang Rongqi Richie }
\IEEEauthorblockA{\textit{Computing Science Joint Degree} \\
Programme, Singapore Institute\\
of Technology - University of\\
Glasgow,\\
2837357W@student.gla.ac.uk}
\and

\IEEEauthorblockN{5\textsuperscript{th} Poon Xiang Yuan }
\IEEEauthorblockA{\textit{Computing Science Joint Degree} \\
Programme, Singapore Institute\\
of Technology - University of\\
Glasgow,\\
2837264P@student.gla.ac.uk}
\and

% \IEEEauthorblockN{2\textsuperscript{nd} }
% \IEEEauthorblockA{\textit{Computing Science Joint Degree} \\
% }
% \and
}

\maketitle


\begin{abstract}
  
Wireless Sensor Networks (WSNs) are pivotal for real-time environmental monitoring applications. This paper delves into the detailed architecture and performance analysis of two prominent protocols, ESP-NOW and LoRa, within the context of a WSN setup. The ESP-NOW protocol, utilizing a Dynamic Source Routing (DSR) mesh algorithm, excels in throughput, latency, and bandwidth compared to LoRa, making it suitable for high-speed, low-latency communication needs. Conversely, LoRa demonstrates advantages in long-range communication and obstacle penetration. 

Additionally, the analysis delves into power consumption, where ESP-NOW outshines LoRa in both idle and active network states. This energy efficiency of ESP-NOW is particularly advantageous for battery-operated devices and remote deployments. 

The findings presented herein provide valuable insights for optimizing protocol selection and power management strategies in WSN deployments, ensuring efficient and reliable data transmission in diverse environmental monitoring scenarios.
\end{abstract}

\begin{IEEEkeywords}
WSN, LoRa, ESP-NOW, Mesh, ELK, MQTT
\end{IEEEkeywords}

\import{sections/}{introduction}

\import{sections/}{lit_review}

\import{sections/}{design}

\import{sections/}{implementation}

\import{sections/}{analysis}

\import{sections/}{conclusion}

\newpage
\printbibliography

\newpage
\appendix
\import{sections/}{appendix}

\end{document}
